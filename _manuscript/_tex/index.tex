\documentclass[
  man,
  longtable,
  nolmodern,
  notxfonts,
  notimes,
  colorlinks=true,linkcolor=blue,citecolor=blue,urlcolor=blue]{apa7}

\usepackage{amsmath}
\usepackage{amssymb}



\usepackage[bidi=default]{babel}
\babelprovide[main,import]{spanish}


% get rid of language-specific shorthands (see #6817):
\let\LanguageShortHands\languageshorthands
\def\languageshorthands#1{}

\RequirePackage{longtable}
\RequirePackage{threeparttablex}

\makeatletter
\renewcommand{\paragraph}{\@startsection{paragraph}{4}{\parindent}%
	{0\baselineskip \@plus 0.2ex \@minus 0.2ex}%
	{-.5em}%
	{\normalfont\normalsize\bfseries\typesectitle}}

\renewcommand{\subparagraph}[1]{\@startsection{subparagraph}{5}{0.5em}%
	{0\baselineskip \@plus 0.2ex \@minus 0.2ex}%
	{-\z@\relax}%
	{\normalfont\normalsize\bfseries\itshape\hspace{\parindent}{#1}\textit{\addperi}}{\relax}}
\makeatother




\usepackage{longtable, booktabs, multirow, multicol, colortbl, hhline, caption, array, float, xpatch}
\setcounter{topnumber}{2}
\setcounter{bottomnumber}{2}
\setcounter{totalnumber}{4}
\renewcommand{\topfraction}{0.85}
\renewcommand{\bottomfraction}{0.85}
\renewcommand{\textfraction}{0.15}
\renewcommand{\floatpagefraction}{0.7}

\usepackage{tcolorbox}
\tcbuselibrary{listings,theorems, breakable, skins}
\usepackage{fontawesome5}

\definecolor{quarto-callout-color}{HTML}{909090}
\definecolor{quarto-callout-note-color}{HTML}{0758E5}
\definecolor{quarto-callout-important-color}{HTML}{CC1914}
\definecolor{quarto-callout-warning-color}{HTML}{EB9113}
\definecolor{quarto-callout-tip-color}{HTML}{00A047}
\definecolor{quarto-callout-caution-color}{HTML}{FC5300}
\definecolor{quarto-callout-color-frame}{HTML}{ACACAC}
\definecolor{quarto-callout-note-color-frame}{HTML}{4582EC}
\definecolor{quarto-callout-important-color-frame}{HTML}{D9534F}
\definecolor{quarto-callout-warning-color-frame}{HTML}{F0AD4E}
\definecolor{quarto-callout-tip-color-frame}{HTML}{02B875}
\definecolor{quarto-callout-caution-color-frame}{HTML}{FD7E14}

%\newlength\Oldarrayrulewidth
%\newlength\Oldtabcolsep


\usepackage{hyperref}




\providecommand{\tightlist}{%
  \setlength{\itemsep}{0pt}\setlength{\parskip}{0pt}}
\usepackage{longtable,booktabs,array}
\usepackage{calc} % for calculating minipage widths
% Correct order of tables after \paragraph or \subparagraph
\usepackage{etoolbox}
\makeatletter
\patchcmd\longtable{\par}{\if@noskipsec\mbox{}\fi\par}{}{}
\makeatother
% Allow footnotes in longtable head/foot
\IfFileExists{footnotehyper.sty}{\usepackage{footnotehyper}}{\usepackage{footnote}}
\makesavenoteenv{longtable}

\usepackage{graphicx}
\makeatletter
\newsavebox\pandoc@box
\newcommand*\pandocbounded[1]{% scales image to fit in text height/width
  \sbox\pandoc@box{#1}%
  \Gscale@div\@tempa{\textheight}{\dimexpr\ht\pandoc@box+\dp\pandoc@box\relax}%
  \Gscale@div\@tempb{\linewidth}{\wd\pandoc@box}%
  \ifdim\@tempb\p@<\@tempa\p@\let\@tempa\@tempb\fi% select the smaller of both
  \ifdim\@tempa\p@<\p@\scalebox{\@tempa}{\usebox\pandoc@box}%
  \else\usebox{\pandoc@box}%
  \fi%
}
% Set default figure placement to htbp
\def\fps@figure{htbp}
\makeatother


% definitions for citeproc citations
\NewDocumentCommand\citeproctext{}{}
\NewDocumentCommand\citeproc{mm}{%
  \begingroup\def\citeproctext{#2}\cite{#1}\endgroup}
\makeatletter
 % allow citations to break across lines
 \let\@cite@ofmt\@firstofone
 % avoid brackets around text for \cite:
 \def\@biblabel#1{}
 \def\@cite#1#2{{#1\if@tempswa , #2\fi}}
\makeatother
\newlength{\cslhangindent}
\setlength{\cslhangindent}{1.5em}
\newlength{\csllabelwidth}
\setlength{\csllabelwidth}{3em}
\newenvironment{CSLReferences}[2] % #1 hanging-indent, #2 entry-spacing
 {\begin{list}{}{%
  \setlength{\itemindent}{0pt}
  \setlength{\leftmargin}{0pt}
  \setlength{\parsep}{0pt}
  % turn on hanging indent if param 1 is 1
  \ifodd #1
   \setlength{\leftmargin}{\cslhangindent}
   \setlength{\itemindent}{-1\cslhangindent}
  \fi
  % set entry spacing
  \setlength{\itemsep}{#2\baselineskip}}}
 {\end{list}}
\usepackage{calc}
\newcommand{\CSLBlock}[1]{\hfill\break\parbox[t]{\linewidth}{\strut\ignorespaces#1\strut}}
\newcommand{\CSLLeftMargin}[1]{\parbox[t]{\csllabelwidth}{\strut#1\strut}}
\newcommand{\CSLRightInline}[1]{\parbox[t]{\linewidth - \csllabelwidth}{\strut#1\strut}}
\newcommand{\CSLIndent}[1]{\hspace{\cslhangindent}#1}





\usepackage{newtx}

\defaultfontfeatures{Scale=MatchLowercase}
\defaultfontfeatures[\rmfamily]{Ligatures=TeX,Scale=1}





\title{Short Paper: A Short Subtitle}


\shorttitle{Short Paper}


\usepackage{etoolbox}






\author{Ian Contreras}



\affiliation{
{Economía y Negocios, Instituto Tecnológico de Santo Domingo}}




\leftheader{Contreras}

\date{2024-11-23}


\abstract{This is the abstract. Lorem ipsum dolor sit amet, consectetur
adipiscing elit. Vestibulum augue turpis, dictum non malesuada a,
volutpat eget velit. Nam placerat turpis purus, eu tristique ex
tincidunt et. Mauris sed augue eget turpis ultrices tincidunt. Sed et mi
in leo porta egestas. Aliquam non laoreet velit. Nunc quis ex vitae eros
aliquet auctor nec ac libero. Duis laoreet sapien eu mi luctus, in
bibendum leo molestie. Sed hendrerit diam diam, ac dapibus nisl volutpat
vitae. Aliquam bibendum varius libero, eu efficitur justo rutrum at. Sed
at tempus elit.}

\keywords{keyword1, keyword2}

\authornote{ 

\par{       }
\par{Correspondence concerning this article should be addressed to Ian
Contreras}
}

\makeatletter
\let\endoldlt\endlongtable
\def\endlongtable{
\hline
\endoldlt
}
\makeatother
\RequirePackage{longtable}
\DeclareDelayedFloatFlavor{longtable}{table}

\urlstyle{same}



\makeatletter
\@ifpackageloaded{caption}{}{\usepackage{caption}}
\AtBeginDocument{%
\ifdefined\contentsname
  \renewcommand*\contentsname{Tabla de contenidos}
\else
  \newcommand\contentsname{Tabla de contenidos}
\fi
\ifdefined\listfigurename
  \renewcommand*\listfigurename{Listado de Figuras}
\else
  \newcommand\listfigurename{Listado de Figuras}
\fi
\ifdefined\listtablename
  \renewcommand*\listtablename{Listado de Tablas}
\else
  \newcommand\listtablename{Listado de Tablas}
\fi
\ifdefined\figurename
  \renewcommand*\figurename{Figura}
\else
  \newcommand\figurename{Figura}
\fi
\ifdefined\tablename
  \renewcommand*\tablename{Tabla}
\else
  \newcommand\tablename{Tabla}
\fi
}
\@ifpackageloaded{float}{}{\usepackage{float}}
\floatstyle{ruled}
\@ifundefined{c@chapter}{\newfloat{codelisting}{h}{lop}}{\newfloat{codelisting}{h}{lop}[chapter]}
\floatname{codelisting}{Listado}
\newcommand*\listoflistings{\listof{codelisting}{Listado de Listados}}
\makeatother
\makeatletter
\makeatother
\makeatletter
\@ifpackageloaded{caption}{}{\usepackage{caption}}
\@ifpackageloaded{subcaption}{}{\usepackage{subcaption}}
\makeatother
\makeatletter
\@ifpackageloaded{fontspec}{}{\usepackage{fontspec}}
\makeatother
\makeatletter
\@ifpackageloaded{draftwatermark}{}{\usepackage{draftwatermark}}
\makeatother
\makeatletter
\@ifpackageloaded{xcolor}{}{\usepackage{xcolor}}
\makeatother
\makeatletter
\@ifpackageloaded{forloop}{}{\usepackage{forloop}}
\makeatother
    \definecolor{watermark}{HTML}{000000}
    
    \newcounter{watermarkrow}
    \newcounter{watermarkcol}

    \DraftwatermarkOptions{
      text={
        \begin{tabular}{c}
          \forloop{watermarkrow}{0}{\value{watermarkrow} < 50}{
            \forloop{watermarkcol}{0}{\value{watermarkcol} < 10}{
              { Borrador}\hspace{4.000000em}
            }
            \\[4.000000em]
          }
        \end{tabular}
      },
      fontsize=1.500000em,
      angle=15.000000,
      color=watermark!20
    }
    

% From https://tex.stackexchange.com/a/645996/211326
%%% apa7 doesn't want to add appendix section titles in the toc
%%% let's make it do it
\makeatletter
\xpatchcmd{\appendix}
  {\par}
  {\addcontentsline{toc}{section}{\@currentlabelname}\par}
  {}{}
\makeatother

%% Disable longtable counter
%% https://tex.stackexchange.com/a/248395/211326

\usepackage{etoolbox}

\makeatletter
\patchcmd{\LT@caption}
  {\bgroup}
  {\bgroup\global\LTpatch@captiontrue}
  {}{}
\patchcmd{\longtable}
  {\par}
  {\par\global\LTpatch@captionfalse}
  {}{}
\apptocmd{\endlongtable}
  {\ifLTpatch@caption\else\addtocounter{table}{-1}\fi}
  {}{}
\newif\ifLTpatch@caption
\makeatother

\begin{document}

\maketitle

\hypertarget{toc}{}
\tableofcontents
\newpage
\section[Introduction]{Short Paper}

\setcounter{secnumdepth}{-\maxdimen} % remove section numbering

\setlength\LTleft{0pt}


\section{Introducción}\label{introducciuxf3n}

\subsection{Planteamiento del
Problema}\label{planteamiento-del-problema}

La coexistencia de dos emisores de deuda gubernamental se ha convertido
en uno de los puntos más cuestionados en la economía dominicana de la
última década. La principal crítica radica en que el Banco Central
realiza una gestión ineficiente de su deuda, reflejada en la evolución
negativa de su patrimonio. No obstante, dado que el Banco Central opera
bajo un régimen de flotación sucia, en el cual la emisión de deuda
funciona como herramienta de control de liquidez (quantitative
tightening), resulta inapropiado evaluar su manejo de pasivos desde una
óptica meramente contable.

El origen de esta dualidad en la emisión de deuda pública se remonta a
la crisis financiera de 2003-2004, cuando el colapso de tres importantes
instituciones bancarias (Baninter, Bancrédito y Banco Mercantil) provocó
una intervención masiva del Banco Central para evitar un riesgo
sistémico. El rescate bancario, que representó aproximadamente el 20.3\%
del PIB, resultó en una emisión de RD\$109,150 millones en instrumentos
de deuda por parte del Banco Central. Esta intervención, si bien
necesaria para mantener la estabilidad del sistema financiero, generó un
deterioro significativo en el balance del banco central y dio origen a
un déficit cuasifiscal que persiste hasta la actualidad, condicionando
tanto el manejo de la política monetaria como fiscal.
(\citeproc{ref-oecd_mercado_2012}{OECD, 2012})

La problemática actual presenta cuatro dimensiones críticas
interrelacionadas. En primer lugar, existen desafíos significativos en
la coordinación de las políticas monetaria y fiscal en un contexto de
vulnerabilidad de la balanza de pagos. República Dominicana, como
importador neto de commodities, enfrenta limitaciones estructurales para
contrarrestar los choques externos en los precios de importación, lo que
genera presiones recurrentes sobre el tipo de cambio y condiciona la
efectividad de la política monetaria. Esta situación se refleja en un
déficit persistente en cuenta corriente, que ha alcanzado niveles
cercanos al 8\% del PIB en años recientes.

En segundo lugar, la competencia entre el Banco Central y el Ministerio
de Hacienda en el mercado de deuda pública ha resultado en un aumento de
los costos de financiamiento, particularmente en el tramo medio de la
curva de tasas de interés. La falta de coordinación entre ambas
instituciones se manifiesta en diferenciales significativos de tasas
para instrumentos de similar vencimiento, llegando a superar los 350
puntos base en algunos casos. Esta divergencia en las tasas de
colocación refleja objetivos institucionales distintos: mientras el
Ministerio de Hacienda busca minimizar el costo del financiamiento
público, el Banco Central utiliza la emisión de deuda como instrumento
de política monetaria para controlar la liquidez y estabilizar el tipo
de cambio. (\citeproc{ref-oecd_mercado_2012}{OECD, 2012})

El tercer aspecto crítico concierne a la evolución del déficit
cuasifiscal y sus implicaciones para las expectativas inflacionarias en
el marco de la curva de Phillips. La acumulación de pérdidas operativas
del Banco Central, que se ha intensificado desde la crisis de 2003-2004,
genera preocupaciones sobre la sostenibilidad de largo plazo de este
esquema de política monetaria. La literatura empírica sugiere que los
déficits cuasifiscales significativos pueden eventualmente resultar en
presiones inflacionarias, sea a través de la monetización directa del
déficit o mediante el deterioro de las expectativas de los agentes
económicos. (\citeproc{ref-cruz-rodriguez_deficit_2006}{Cruz-Rodríguez,
2006})

Un cuarto elemento crítico radica en las limitaciones constitucionales
que enfrenta el Banco Central para financiar al gobierno central en el
mercado de bonos. Esta restricción, que solo puede ser levantada en
circunstancias excepcionales y bajo condiciones estrictas, reduce
significativamente los instrumentos disponibles para el control de la
depreciación del tipo de cambio.

El debate sobre la reforma de este esquema institucional ha generado dos
posiciones principales. Por un lado, algunos economistas, abogan por la
consolidación de la deuda gubernamental bajo el Ministerio de Hacienda,
argumentando que la gestión actual ha sido ineficiente y ha resultado en
un crecimiento significativo de la deuda del Banco Central, que
actualmente representa aproximadamente el 15\% del PIB. Esta posición
sugiere que la centralización del manejo de la deuda en un solo emisor
soberano permitiría mejorar las condiciones de emisión y reducir los
costos de financiamiento.

La posición contraria sostiene que, dada la diferente naturaleza de las
funciones de reacción de ambas instituciones, la unificación de la
emisión de deuda podría ser contraproducente para el logro de sus
respectivos objetivos. Además, argumentan que el Banco Central perdería
su instrumento más efectivo para controlar las presiones sobre el tipo
de cambio, lo cual podría comprometer la estabilidad macroeconómica en
un contexto de alta vulnerabilidad externa.

La determinación del régimen óptimo de coordinación económica representa
uno de los mayores desafíos de política económica que enfrenta la
República Dominicana actualmente. Esta problemática presenta dos
dimensiones fundamentales: por un lado, la necesidad de establecer
mecanismos efectivos de coordinación entre las autoridades monetarias y
financieras, y por otro, la identificación de los instrumentos más
adecuados para alcanzar los objetivos de política. La complejidad de
este reto radica en la necesidad de adoptar una perspectiva
multidimensional que considere los diferentes equilibrios económicos
implicados. El mecanismo actual de control de liquidez presenta un
dilema fundamental: ¿es preferible mantener una política con mayor costo
económico que ha demostrado efectividad en el control de las
expectativas cambiarias, o este enfoque representa un riesgo
insostenible para las finanzas del Banco Central en el largo plazo? La
resolución de esta disyuntiva requiere la consideración de múltiples
factores, incluyendo la prima de riesgo por intervención de liquidez, la
competencia en los tramos corto y medio de la curva de intereses, y las
implicaciones de la ``aritmética monetarista desagradable''. Como
resultado, la búsqueda de un marco institucional óptimo que balancee
estos diferentes objetivos y restricciones se mantiene como una pregunta
abierta en el diseño de la política económica dominicana.

\subsection{Propósito de la
Investigación}\label{propuxf3sito-de-la-investigaciuxf3n}

Este estudio contribuye a la literatura de política económica mediante
el análisis de los regímenes de coordinación óptimos entre la política
monetaria y fiscal en la República Dominicana, empleando un marco
teórico de juegos para una economía pequeña y abierta. La investigación
desarrolla un modelo de equilibrio general dinámico y estocástico (DSGE)
que incorpora las características fundamentales de la economía
dominicana, incluyendo una restricción intertemporal consolidada para
las instituciones gubernamentales y un mecanismo de transmisión del tipo
de cambio a la inflación. Este enfoque permite examinar las
interacciones estratégicas entre las autoridades monetarias y fiscales
bajo diferentes escenarios de coordinación, contribuyendo así a la
comprensión de las implicaciones del déficit cuasifiscal del Banco
Central.

La metodología propuesta se distingue por tres aspectos innovadores.
Primero, desarrolla un análisis basado en teoría de juegos del esquema
líder-seguidor entre las políticas monetaria y fiscal, específicamente
adaptado para una economía pequeña y abierta como la dominicana.
Segundo, implementa un conjunto de experimentos contrafactuales
utilizando parámetros estructurales del período 2012-2024, que
permitirán evaluar diferentes regímenes de coordinación y sus
implicaciones para el bienestar social. Tercero, incorpora un análisis
de sensibilidad global que examina la robustez de los resultados ante
variaciones en los parámetros estructurales del modelo, proporcionando
así una evaluación comprehensiva de la estabilidad y unicidad de los
equilibrios encontrados. Esta aproximación metodológica permite
identificar el régimen de coordinación que minimiza las pérdidas
sociales, ofreciendo recomendaciones de política económica basadas en
evidencia cuantitativa.

\subsection{Preguntas de
Investigación}\label{preguntas-de-investigaciuxf3n}

\begin{enumerate}
\def\labelenumi{\arabic{enumi}.}
\item
  ¿Cuál es el esquema óptimo de coordinación de políticas fiscales y
  monetarias en el contexto de la economía dominicana?
\item
  ¿Cuál es el dominio de los parámetros estructurales de forma que se
  asegura la estabilidad del regimen de coordinación óptimo?
\item
  ¿Qué reformas y/o instrumentos alternativos podrían ser necesarios
  para alcanzar el esquema de coordinación óptimo previamente definido?
\end{enumerate}

\newpage{}

\section{Revisión Literaria}\label{revisiuxf3n-literaria}

\subsection{Marco teórico}\label{marco-teuxf3rico}

El análisis de la política monetaria se ha centrado históricamente en
responder dos preguntas fundamentales: los impactos de una regla de
política monetaria sobre la economía y la regla de política monetaria
óptima bajo ciertas condiciones. Esta segunda interrogante ha cobrado
especial relevancia en el contexto de las reglas de la adopción de los
esquemas de metas explícitas de inflación.

La discusión sobre reglas versus discrecionalidad en política monetaria
tiene sus raíces en los trabajos seminales de
(\citeproc{ref-kydland_rules_1977}{Kydland \& Prescott, 1977}) y
(\citeproc{ref-barro_rules_1983}{Barro \& Gordon, 1983}), quienes
demostraron que la extrema flexibilidad o discrecionalidad en el manejo
monetario conduce a una pérdida de credibilidad, afectando directamente
el objetivo de estabilización y generando un ``sesgo inflacionario''.
Este hallazgo fundamental llevó al desarrollo de un marco analítico
basado en reglas monetarias que pudieran servir como guía para las
intervenciones de los bancos centrales.

(\citeproc{ref-taylor_discretion_1993}{Taylor, 1993}) marcó un hito en
esta literatura al proponer una regla simple que relaciona la tasa de
interés nominal con las desviaciones de la inflación respecto a su nivel
objetivo y las desviaciones del producto respecto a su nivel potencial.
Su trabajo demostró que las políticas monetarias centradas en estos
objetivos son más eficientes que aquellas que se ajustan con la oferta
de dinero y el tipo de cambio.
(\citeproc{ref-svensson_inflation_1997}{Svensson, 1997})y
(\citeproc{ref-clarida_science_1999}{Clarida et~al., 1999})
posteriormente formalizaron los fundamentos teóricos de la regla de
Taylor y propusieron diversas modificaciones sobre esta base de
referencia.

La interacción entre políticas monetaria y fiscal introduce una
dimensión adicional al análisis.
(\citeproc{ref-nordhaus_policy_1994}{Nordhaus, 1994}) examinó esta
interacción desde un marco de teoría de juegos, encontrando que la falta
de coordinación entre autoridades puede resultar en equilibrios
subóptimos caracterizados por altas tasas de interés y déficits
presupuestarios. (\citeproc{ref-van_aarle_monetary_1995}{Aarle et~al.,
1995}) profundizaron este análisis en el contexto de la estabilización
de la deuda, mientras que estudiaron la necesidad de ajuste de la
política monetaria para mantener baja inflación en el marco de distintas
estructuras de coordinación entre hacedores de política.

El desarrollo de modelos de equilibrio general dinámico y estocástico
(DSGE) ha proporcionado un marco más riguroso para analizar estas
interacciones. El modelo seminal de
(\citeproc{ref-gali_monetary_2005}{Galí \& Monacelli, 2005}) para
economías pequeñas y abiertas incorpora rigideces nominales y
competencia monopolística en un marco de optimización intertemporal. La
estructura del modelo incluye hogares que maximizan su utilidad,
empresas que operan en un entorno de competencia monopolística con
precios rígidos siguiendo el esquema de
(\citeproc{ref-calvo_staggered_1983}{Calvo, 1983}), y autoridades
monetarias y fiscales que implementan políticas económicas.

Este marco teórico permite analizar la transmisión de la política
monetaria a través de varios canales. La rigidez de precios de Calvo
genera efectos reales de la política monetaria y captura la persistencia
observada en la inflación. La apertura de la economía se modela mediante
un índice de precios al consumidor que incluye tanto bienes domésticos
como importados, con un mecanismo de pass-through del tipo de cambio a
la inflación, aspecto crucial para economías pequeñas y abiertas.

La literatura más reciente, ejemplificada por trabajos como
(\citeproc{ref-bartolomeo_fiscal-monetary_2005}{Bartolomeo \&
Gioacchino, 2005}), ha integrado la teoría de juegos en este marco DSGE
para analizar las interacciones estratégicas entre autoridades
monetarias y fiscales. Este enfoque permite examinar diferentes
escenarios de coordinación, desde equilibrios Nash simultáneos hasta
soluciones Stackelberg con distintos ordenamientos de liderazgo,
proporcionando insights sobre los regímenes de coordinación óptimos para
diferentes estructuras económicas.

\subsection{Antecedentes}\label{antecedentes}

La literatura sobre coordinación de políticas económicas en la República
Dominicana presenta un vacío significativo en cuanto al análisis de
interacciones estratégicas entre autoridades monetaria y fiscal mediante
teoría de juegos en un contexto de modelos DSGE. El referente
internacional más cercano a este tipo de análisis es el trabajo de
(\citeproc{ref-tetik_evaluation_2021}{Tetik \& Ceylan, 2021}) para la
economía turca, que comparte características similares como economía
pequeña y abierta. Esta brecha en la literatura dominicana representa
una oportunidad importante para expandir nuestra comprensión de los
mecanismos de coordinación de políticas económicas en el país.

(\citeproc{ref-tetik_evaluation_2021}{Tetik \& Ceylan, 2021}) aplica
teoría de juegos al análisis de la interacción entre autoridades
monetaria y fiscal sobre el modelo neokeynesiano de Galí. Mediante
experimentos contrafactuales con datos de la economía turca para
2006-2019, encuentra que el escenario que minimiza la pérdida social
ocurre cuando la autoridad monetaria actúa como líder de Stackelberg.
Sus resultados muestran patrones de respuesta específicos ante choques
exógenos, que varían según la naturaleza del choque y la posición
jerárquica de cada autoridad.

(\citeproc{ref-ramirez_modelo_2014}{Ramírez \& Torres, 2014}) realiza la
primera publicación de una estimación de un modelo DSGE para la
República Dominicana basado en el marco de
(\citeproc{ref-gali_monetary_2005}{Galí \& Monacelli, 2005}). Su modelo
incorpora fricciones nominales y persistencia en hábitos de consumo
característicos de la economía dominicana. La estimación bayesiana de
parámetros estructurales establece una base cuantitativa para el
análisis de política monetaria en el país.

(\citeproc{ref-perez_perez_nueva_2021}{Pérez Pérez, 2021}) determina
reglas de política monetaria óptimas para la República Dominicana
comparando especificaciones de curvas de reacción. Su investigación
indica que una regla de política monetaria forward-looking que considera
la inflación observada, la inflación proyectada y la inercia de la tasa
de interés genera mejores resultados de bienestar social, con
variaciones según los choques que afectan a la economía.

La presente investigación se construye sobre estos antecedentes de
manera integral, combinando el marco metodológico de teoría de juegos
desarrollado por Tetik con las estimaciones de parámetros estructurales
de Ramírez y los hallazgos sobre reglas monetarias óptimas de Pérez.
Esta síntesis permite desarrollar un marco analítico robusto que examina
por primera vez las interacciones estratégicas entre autoridades
monetaria y fiscal en la República Dominicana.

\newpage{}

\section{Estructura del Modelo}\label{estructura-del-modelo}

\subsection{Un modelo DSGE neokeynesiano para la economía
dominicana}\label{un-modelo-dsge-neokeynesiano-para-la-economuxeda-dominicana}

Basándonos en los precedentes planteados por
(\citeproc{ref-ramirez_modelo_2014}{Ramírez \& Torres, 2014}) y
(\citeproc{ref-perez_perez_nueva_2021}{Pérez Pérez, 2021}), la economía
dominicana se puede caracterizar de manera adecuada por un modelo de
equilibrio general neokeynesiano de una economía pequeña y abierta que
incorpora fricciones financieras en el mercado de bonos gubernamentales,
planteado en (\citeproc{ref-gali_monetary_2015}{Galí, 2015}). El modelo
está habitado por cuatro agentes económicos principales que optimizan
sus decisiones: los hogares, que enfrentan costos de ajuste de
portafolio al distribuir sus recursos entre bonos de corto y largo
plazo; las firmas domésticas que operan en competencia monopolística;
los importadores; y el gobierno como autoridad de política fiscal y el
Banco Central como la autoridad monetaria. La característica distintiva
del modelo es la incorporación de rigideces nominales de precios tipo
Calvo, impuestos distorsionadores, y un pass-through completo del tipo
de cambio, donde la demanda agregada depende de un promedio ponderado de
las tasas de interés de corto y largo plazo.

La estructura del modelo incorpora diversos canales de transmisión y
fuentes de choques, donde la curva de Phillips caracteriza una oferta
agregada con precios rígidos que no se ajustan instantáneamente a
cambios en costos o demanda. El mecanismo de transmisión internacional
opera a través del tipo de cambio nominal y la paridad descubierta de
tasas de interés, permitiendo que los choques externos afecten a la
economía doméstica. El modelo considera cinco fuentes principales de
incertidumbre: choques de política monetaria, productividad,
preferencias, tasa de interés internacional y nivel de precios mundial.
La regla de política monetaria óptima se deriva de al comparar los
equilibrios generales resultantes de las intereacciónes entre el Banco
Central y el Ministerio de Hacienda en los distintos regímenes de
coordinación.

\subsection{Derivación de las curvas de reacción bajo un regimen de
independencia
total.}\label{derivaciuxf3n-de-las-curvas-de-reacciuxf3n-bajo-un-regimen-de-independencia-total.}

\subsection{Derivación de las curvas de reacción bajo un regimen de
dominancia
fiscal}\label{derivaciuxf3n-de-las-curvas-de-reacciuxf3n-bajo-un-regimen-de-dominancia-fiscal}

\subsection{Derivación de la curva de reacción bajo un regimen de
dominancia
monetaria}\label{derivaciuxf3n-de-la-curva-de-reacciuxf3n-bajo-un-regimen-de-dominancia-monetaria}

\subsection{Calibración paramétrica para la simulación
dinámica}\label{calibraciuxf3n-paramuxe9trica-para-la-simulaciuxf3n-dinuxe1mica}

\section{Resultados}\label{resultados}

\subsection{Análisis de perdidas sociales: Determinación del regimen de
interacción
óptimo}\label{anuxe1lisis-de-perdidas-sociales-determinaciuxf3n-del-regimen-de-interacciuxf3n-uxf3ptimo}

\subsection{Análisis de los choques estructurales: Función de
impulso-reacción}\label{anuxe1lisis-de-los-choques-estructurales-funciuxf3n-de-impulso-reacciuxf3n}

\subsection{Análisis de sensibilidad: Evaluación de la estabilidad
paramétrica por filtro de Monte
Carlo}\label{anuxe1lisis-de-sensibilidad-evaluaciuxf3n-de-la-estabilidad-paramuxe9trica-por-filtro-de-monte-carlo}

\section{Conclusiones y
Recomendaciones}\label{conclusiones-y-recomendaciones}

\newpage{}

\section{Referencias}\label{referencias}

References marked with an asterisk indicate studies included in the
meta-analysis.

\phantomsection\label{refs}
\begin{CSLReferences}{1}{0}
\bibitem[\citeproctext]{ref-van_aarle_monetary_1995}
Aarle, B. van, Bovenberg, A. L., \& Raith, M. (1995). Monetary and
fiscal policy interaction and government debt stabilization.
\emph{Discussion Paper}.
\url{https://ideas.repec.org//p/tiu/tiucen/3e0859f2-375c-4cf5-bf90-4555884390b7.html}

\bibitem[\citeproctext]{ref-adjemian_dynare_2011}
Adjemian, S., Bastani, H., Juillard, M., Karamé, F., Maih, J., Mihoubi,
F., Perendia, G., Pfeifer, J., Ratto, M., \& Villemot, S. (2011).
\emph{Dynare: Reference Manual, Version 4}. {CEPREMAP}.

\bibitem[\citeproctext]{ref-aliaga_miranda_monetary_2020}
Aliaga Miranda, A. (2020). Monetary policy rules for an open economy
with financial frictions: A Bayesian approach. \emph{Dynare Working
Papers}. \url{https://ideas.repec.org//p/cpm/dynare/062.html}

\bibitem[\citeproctext]{ref-barro_rules_1983}
Barro, R. J., \& Gordon, D. B. (1983). \emph{Rules, Discretion and
Reputation in a Model of Monetary Policy} (1079). National Bureau of
Economic Research. \url{https://doi.org/10.3386/w1079}

\bibitem[\citeproctext]{ref-bartolomeo_fiscal-monetary_2005}
Bartolomeo, G. D., \& Gioacchino, D. D. (2005). \emph{Fiscal-Monetary
Policy Coordination And Debt Management: A Two Stage Dynamic Analysis}
(Macroeconomics 0504024). University Library of Munich, Germany.
\url{https://ideas.repec.org/p/wpa/wuwpma/0504024.html}

\bibitem[\citeproctext]{ref-calvo_staggered_1983}
Calvo, G. A. (1983). Staggered prices in a utility-maximizing framework.
\emph{Journal of Monetary Economics}, \emph{12}(3), 383-398.
\url{https://doi.org/10.1016/0304-3932(83)90060-0}

\bibitem[\citeproctext]{ref-clarida_science_1999}
Clarida, R., Gali, J., \& Gertler, M. (1999). The Science of Monetary
Policy: A New Keynesian Perspective. \emph{Journal of Economic
Literature}, \emph{37}(4), 1661-1707.
\url{https://doi.org/10.1257/jel.37.4.1661}

\bibitem[\citeproctext]{ref-cruz-rodriguez_deficit_2006}
Cruz-Rodríguez, A. (2006, junio 20). \emph{El déficit cuasifiscal del
Banco Central de la República Dominicana} {[}\{MPRA\} Paper{]}.
\url{https://mpra.ub.uni-muenchen.de/109191/}

\bibitem[\citeproctext]{ref-gali_inflation_2000}
Gali, J., \& Gertler, M. (2000). \emph{Inflation Dynamics: A Structural
Econometric Analysis} (7551). National Bureau of Economic Research.
\url{https://doi.org/10.3386/w7551}

\bibitem[\citeproctext]{ref-gali_monetary_2015}
Galí, J. (2015). Monetary Policy, Inflation, and the Business Cycle: An
Introduction to the New Keynesian Framework and Its Applications Second
edition. \emph{Economics Books}.
\url{https://ideas.repec.org//b/pup/pbooks/10495.html}

\bibitem[\citeproctext]{ref-gali_monetary_2005}
Galí, J., \& Monacelli, T. (2005). Monetary Policy and Exchange Rate
Volatility in a Small Open Economy. \emph{The Review of Economic
Studies}, \emph{72}(3), 707-734.
\url{https://ideas.repec.org/a/oup/restud/v72y2005i3p707-734.html}

\bibitem[\citeproctext]{ref-gobierno_de_la_republica_dominicana_ley_nodate}
Gobierno de la República Dominicana. (s.~f.). \emph{Ley No. 183-02
Monetaria y Financiera}. Recuperado 23 de noviembre de 2024, de
\url{https://www.sb.gob.do/regulacion/compendio-de-leyes-y-reglamentos/ley-no-183-02-monetaria-y-financiera/}

\bibitem[\citeproctext]{ref-hallett_monetary_2014}
HALLETT, A. H., LIBICH, J., \& STEHLÍK, P. (2014). Monetary and Fiscal
Policy Interaction with Various Degrees of Commitment. \emph{Czech
Journal of Economics and Finance (Finance a uver)}, \emph{64}(1), 2-29.
\url{https://ideas.repec.org/a/fau/fauart/v64y2014i1p2-29.html}

\bibitem[\citeproctext]{ref-henry_central_2003}
Henry, B., Nixon, J., \& Hall, S. (2003). Central Bank Independence and
Co‐ordinating Monetary and Fiscal Policy. \emph{Economic Outlook},
\emph{23}, 7-13. \url{https://doi.org/10.1111/1468-0319.00162}

\bibitem[\citeproctext]{ref-hidalgo_economista_2024}
Hidalgo, M. (2024, septiembre 9). \emph{Economista propone transferir
deuda del Banco Central al Ministerio de Hacienda. El Nuevo Diario
(República Dominicana)}.
\url{https://elnuevodiario.com.do/economista-propone-transferir-deuda-del-banco-central-al-ministerio-de-hacienda/}

\bibitem[\citeproctext]{ref-hinterlang_classification_2022}
Hinterlang, N., \& Hollmayr, J. (2022). Classification of monetary and
fiscal dominance regimes using machine learning techniques.
\emph{Journal of Macroeconomics}, \emph{74}, 103469.
https://doi.org/\url{https://doi.org/10.1016/j.jmacro.2022.103469}

\bibitem[\citeproctext]{ref-kleineman_central_2021}
Kleineman, J. (Ed.). (2021). \emph{Central Bank Independence: The
Economic Foundations, the Constitutional Implications and Democratic
Accountability}. Brill Nijhoff.
\url{https://brill.com/edcollbook/title/10804}

\bibitem[\citeproctext]{ref-kydland_rules_1977}
Kydland, F. E., \& Prescott, E. C. (1977). Rules Rather than Discretion:
The Inconsistency of Optimal Plans. \emph{Journal of Political Economy},
\emph{85}(3), 473-491. \url{https://www.jstor.org/stable/1830193}

\bibitem[\citeproctext]{ref-malovana_foreign_2015}
Malovana, S. (2015). Foreign Exchange Interventions at the Zero Lower
Bound in the Czech Economy: A {DSGE} Approach. \emph{Working Papers
{IES}}. \url{https://ideas.repec.org//p/fau/wpaper/wp2015_13.html}

\bibitem[\citeproctext]{ref-maskedreference}
Masked Citation. (n.d.). \emph{Masked Title}.

\bibitem[\citeproctext]{ref-nordhaus_policy_1994}
Nordhaus, W. (1994). Policy games: Coordination and Independece in
Monetary and Fiscal Policies. \emph{Brookings Papers on Economic
Activity}, \emph{25}(2), 139-216.
\url{https://econpapers.repec.org/article/binbpeajo/v_3a25_3ay_3a1994_3ai_3a1994-2_3ap_3a139-216.htm}

\bibitem[\citeproctext]{ref-oecd_mercado_2012}
OECD. (2012). \emph{El mercado de capitales en República Dominicana:
Aprovechando su potencial para el desarrollo}. Organisation for Economic
Co-operation; Development.
\url{https://www.oecd-ilibrary.org/fr/development/el-mercado-de-capitales-en-republica-dominicana_9789264177680-es}

\bibitem[\citeproctext]{ref-perez_perez_nueva_2021}
Pérez Pérez, M. A. (2021). \emph{Nueva literatura económica dominicana:
premios del Concurso de Economía «Biblioteca Juan Pablo Duarte» 2020}
(J. Alcántara Almánzar, E. F. Soto, F. A. Pérez Quiñones, I. Miolán, \&
H. Batista, Eds.). Banco Central de la República Dominicana ({BCRD}).

\bibitem[\citeproctext]{ref-petit_fiscal_1989}
Petit, M. L. (1989). Fiscal and Monetary Policy Co-Ordination: A
Differential Game Approach. \emph{Journal of Applied Econometrics},
\emph{4}(2), 161-179. \url{https://www.jstor.org/stable/2096467}

\bibitem[\citeproctext]{ref-ramirez_modelo_2014}
Ramírez, F. A., \& Torres, F. (2014). Modelo de Equilibrio General
Dinámico y Estocástico con Rigideces Nominales para el Análisis de
Política y Proyecciones en la República Dominicana. \emph{Foro de
Investigadores de Bancos Centrales del Consejo Monetario
Centroamericano}.
\url{https://www.secmca.org/recard/index.php/foro/article/view/69}

\bibitem[\citeproctext]{ref-sandstrom_inflation_2022}
Sandström, C. (2022). \emph{Inflation and Quantitative Tightening - A
theoretical assessment of contractionary monetary policy and real
economic activity}.
\url{http://lup.lub.lu.se/student-papers/record/9096134}

\bibitem[\citeproctext]{ref-saulo_fiscal_2013}
Saulo, H., Rêgo, L. C., \& Divino, J. A. (2013). Fiscal and monetary
policy interactions: a game theory approach. \emph{Annals of Operations
Research}, \emph{206}(1), 341-366.
\url{https://doi.org/10.1007/s10479-013-1379-3}

\bibitem[\citeproctext]{ref-svensson_inflation_1997}
Svensson, L. E. O. (1997). Inflation forecast targeting: Implementing
and monitoring inflation targets. \emph{European Economic Review},
\emph{41}(6), 1111-1146.
\url{https://ideas.repec.org//a/eee/eecrev/v41y1997i6p1111-1146.html}

\bibitem[\citeproctext]{ref-taylor_discretion_1993}
Taylor, J. B. (1993). Discretion versus policy rules in practice.
\emph{Carnegie-Rochester Conference Series on Public Policy},
\emph{39}(1), 195-214.
\url{https://ideas.repec.org//a/eee/crcspp/v39y1993ip195-214.html}

\bibitem[\citeproctext]{ref-tetik_evaluation_2021}
Tetik, M., \& Ceylan, R. (2021). Evaluation of Stackelberg
Leader-Follower Interaction Between Policymakers in Small-Scale Open
Economies*. \emph{Ekonomika}, \emph{100}(2), 101-132.
\url{https://www.redalyc.org/journal/6922/692272891005/html/}

\end{CSLReferences}






\end{document}
