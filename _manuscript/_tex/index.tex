% Options for packages loaded elsewhere
\PassOptionsToPackage{unicode}{hyperref}
\PassOptionsToPackage{hyphens}{url}
\PassOptionsToPackage{dvipsnames,svgnames,x11names}{xcolor}
%
\documentclass[
  authoryear,
  preprint]{elsarticle}

\usepackage{amsmath,amssymb}
\usepackage{iftex}
\ifPDFTeX
  \usepackage[T1]{fontenc}
  \usepackage[utf8]{inputenc}
  \usepackage{textcomp} % provide euro and other symbols
\else % if luatex or xetex
  \usepackage{unicode-math}
  \defaultfontfeatures{Scale=MatchLowercase}
  \defaultfontfeatures[\rmfamily]{Ligatures=TeX,Scale=1}
\fi
\usepackage{lmodern}
\ifPDFTeX\else  
    % xetex/luatex font selection
\fi
% Use upquote if available, for straight quotes in verbatim environments
\IfFileExists{upquote.sty}{\usepackage{upquote}}{}
\IfFileExists{microtype.sty}{% use microtype if available
  \usepackage[]{microtype}
  \UseMicrotypeSet[protrusion]{basicmath} % disable protrusion for tt fonts
}{}
\makeatletter
\@ifundefined{KOMAClassName}{% if non-KOMA class
  \IfFileExists{parskip.sty}{%
    \usepackage{parskip}
  }{% else
    \setlength{\parindent}{0pt}
    \setlength{\parskip}{6pt plus 2pt minus 1pt}}
}{% if KOMA class
  \KOMAoptions{parskip=half}}
\makeatother
\usepackage{xcolor}
\setlength{\emergencystretch}{3em} % prevent overfull lines
\setcounter{secnumdepth}{5}
% Make \paragraph and \subparagraph free-standing
\makeatletter
\ifx\paragraph\undefined\else
  \let\oldparagraph\paragraph
  \renewcommand{\paragraph}{
    \@ifstar
      \xxxParagraphStar
      \xxxParagraphNoStar
  }
  \newcommand{\xxxParagraphStar}[1]{\oldparagraph*{#1}\mbox{}}
  \newcommand{\xxxParagraphNoStar}[1]{\oldparagraph{#1}\mbox{}}
\fi
\ifx\subparagraph\undefined\else
  \let\oldsubparagraph\subparagraph
  \renewcommand{\subparagraph}{
    \@ifstar
      \xxxSubParagraphStar
      \xxxSubParagraphNoStar
  }
  \newcommand{\xxxSubParagraphStar}[1]{\oldsubparagraph*{#1}\mbox{}}
  \newcommand{\xxxSubParagraphNoStar}[1]{\oldsubparagraph{#1}\mbox{}}
\fi
\makeatother


\providecommand{\tightlist}{%
  \setlength{\itemsep}{0pt}\setlength{\parskip}{0pt}}\usepackage{longtable,booktabs,array}
\usepackage{calc} % for calculating minipage widths
% Correct order of tables after \paragraph or \subparagraph
\usepackage{etoolbox}
\makeatletter
\patchcmd\longtable{\par}{\if@noskipsec\mbox{}\fi\par}{}{}
\makeatother
% Allow footnotes in longtable head/foot
\IfFileExists{footnotehyper.sty}{\usepackage{footnotehyper}}{\usepackage{footnote}}
\makesavenoteenv{longtable}
\usepackage{graphicx}
\makeatletter
\newsavebox\pandoc@box
\newcommand*\pandocbounded[1]{% scales image to fit in text height/width
  \sbox\pandoc@box{#1}%
  \Gscale@div\@tempa{\textheight}{\dimexpr\ht\pandoc@box+\dp\pandoc@box\relax}%
  \Gscale@div\@tempb{\linewidth}{\wd\pandoc@box}%
  \ifdim\@tempb\p@<\@tempa\p@\let\@tempa\@tempb\fi% select the smaller of both
  \ifdim\@tempa\p@<\p@\scalebox{\@tempa}{\usebox\pandoc@box}%
  \else\usebox{\pandoc@box}%
  \fi%
}
% Set default figure placement to htbp
\def\fps@figure{htbp}
\makeatother

\makeatletter
\@ifpackageloaded{caption}{}{\usepackage{caption}}
\AtBeginDocument{%
\ifdefined\contentsname
  \renewcommand*\contentsname{Tabla de contenidos}
\else
  \newcommand\contentsname{Tabla de contenidos}
\fi
\ifdefined\listfigurename
  \renewcommand*\listfigurename{Listado de Figuras}
\else
  \newcommand\listfigurename{Listado de Figuras}
\fi
\ifdefined\listtablename
  \renewcommand*\listtablename{Listado de Tablas}
\else
  \newcommand\listtablename{Listado de Tablas}
\fi
\ifdefined\figurename
  \renewcommand*\figurename{Figura}
\else
  \newcommand\figurename{Figura}
\fi
\ifdefined\tablename
  \renewcommand*\tablename{Tabla}
\else
  \newcommand\tablename{Tabla}
\fi
}
\@ifpackageloaded{float}{}{\usepackage{float}}
\floatstyle{ruled}
\@ifundefined{c@chapter}{\newfloat{codelisting}{h}{lop}}{\newfloat{codelisting}{h}{lop}[chapter]}
\floatname{codelisting}{Listado}
\newcommand*\listoflistings{\listof{codelisting}{Listado de Listados}}
\makeatother
\makeatletter
\makeatother
\makeatletter
\@ifpackageloaded{caption}{}{\usepackage{caption}}
\@ifpackageloaded{subcaption}{}{\usepackage{subcaption}}
\makeatother
\journal{Journal Name}
\makeatletter
\@ifpackageloaded{fontspec}{}{\usepackage{fontspec}}
\makeatother
\makeatletter
\@ifpackageloaded{draftwatermark}{}{\usepackage{draftwatermark}}
\makeatother
\makeatletter
\@ifpackageloaded{xcolor}{}{\usepackage{xcolor}}
\makeatother
\makeatletter
\@ifpackageloaded{forloop}{}{\usepackage{forloop}}
\makeatother
    \definecolor{watermark}{HTML}{000000}
    
    \newcounter{watermarkrow}
    \newcounter{watermarkcol}

    \DraftwatermarkOptions{
      text={
        \begin{tabular}{c}
          \forloop{watermarkrow}{0}{\value{watermarkrow} < 50}{
            \forloop{watermarkcol}{0}{\value{watermarkcol} < 10}{
              { Borrador}\hspace{4.000000em}
            }
            \\[4.000000em]
          }
        \end{tabular}
      },
      fontsize=1.500000em,
      angle=15.000000,
      color=watermark!20
    }
    

\ifLuaTeX
\usepackage[bidi=basic]{babel}
\else
\usepackage[bidi=default]{babel}
\fi
\babelprovide[main,import]{spanish}
% get rid of language-specific shorthands (see #6817):
\let\LanguageShortHands\languageshorthands
\def\languageshorthands#1{}
\usepackage[]{natbib}
\bibliographystyle{elsarticle-harv}
\nocite{*}
\usepackage{bookmark}

\IfFileExists{xurl.sty}{\usepackage{xurl}}{} % add URL line breaks if available
\urlstyle{same} % disable monospaced font for URLs
\hypersetup{
  pdftitle={Short Paper},
  pdfauthor={Ian Contreras},
  pdflang={es},
  pdfkeywords={keyword1, keyword2},
  colorlinks=true,
  linkcolor={blue},
  filecolor={Maroon},
  citecolor={Blue},
  urlcolor={Blue},
  pdfcreator={LaTeX via pandoc}}


\setlength{\parindent}{6pt}
\begin{document}

\begin{frontmatter}
\title{Short Paper \\\large{A Short Subtitle} }
\author[1]{Ian Contreras%
\corref{cor1}%
\fnref{fn1}}


\affiliation[1]{organization={Instituto Tecnológico de Santo
Domingo, Economía y Negocios},city={Santo Domingo},country={República
Dominicana},countrysep={,},postcodesep={}}

\cortext[cor1]{Corresponding author}
\fntext[fn1]{Lorem Ipsum}
        
\begin{abstract}
This is the abstract. Lorem ipsum dolor sit amet, consectetur adipiscing
elit. Vestibulum augue turpis, dictum non malesuada a, volutpat eget
velit. Nam placerat turpis purus, eu tristique ex tincidunt et. Mauris
sed augue eget turpis ultrices tincidunt. Sed et mi in leo porta
egestas. Aliquam non laoreet velit. Nunc quis ex vitae eros aliquet
auctor nec ac libero. Duis laoreet sapien eu mi luctus, in bibendum leo
molestie. Sed hendrerit diam diam, ac dapibus nisl volutpat vitae.
Aliquam bibendum varius libero, eu efficitur justo rutrum at. Sed at
tempus elit.
\end{abstract}





\begin{keyword}
    keyword1 \sep 
    keyword2
\end{keyword}
\end{frontmatter}
    
\renewcommand*\contentsname{Tabla de contenidos}
{
\hypersetup{linkcolor=}
\setcounter{tocdepth}{3}
\tableofcontents
}

\section{Introducción}\label{introducciuxf3n}

\subsection{Planteamiento del
Problema}\label{planteamiento-del-problema}

La coexistencia de dos emisores de deuda gubernamental se ha convertido
en uno de los puntos más cuestionados en la economía dominicana de la
última década. La principal crítica radica en que el Banco Central
realiza una gestión ineficiente de su deuda, reflejada en la evolución
negativa de su patrimonio. No obstante, dado que el Banco Central opera
bajo un régimen de flotación sucia, en el cual la emisión de deuda
funciona como herramienta de control de liquidez (quantitative
tightening), resulta inapropiado evaluar su manejo de pasivos desde una
óptica meramente contable.

El origen de esta dualidad en la emisión de deuda pública se remonta a
la crisis financiera de 2003-2004, cuando el colapso de tres importantes
instituciones bancarias (Baninter, Bancrédito y Banco Mercantil) provocó
una intervención masiva del Banco Central para evitar un riesgo
sistémico. El rescate bancario, que representó aproximadamente el 20.3\%
del PIB, resultó en una emisión de RD\$109,150 millones en instrumentos
de deuda por parte del Banco Central. Esta intervención, si bien
necesaria para mantener la estabilidad del sistema financiero, generó un
deterioro significativo en el balance del banco central y dio origen a
un déficit cuasifiscal que persiste hasta la actualidad, condicionando
tanto el manejo de la política monetaria como fiscal.
\citep{oecd_mercado_2012}

La problemática actual presenta cuatro dimensiones críticas
interrelacionadas. En primer lugar, existen desafíos significativos en
la coordinación de las políticas monetaria y fiscal en un contexto de
vulnerabilidad de la balanza de pagos. República Dominicana, como
importador neto de commodities, enfrenta limitaciones estructurales para
contrarrestar los choques externos en los precios de importación, lo que
genera presiones recurrentes sobre el tipo de cambio y condiciona la
efectividad de la política monetaria. Esta situación se refleja en un
déficit persistente en cuenta corriente, que ha alcanzado niveles
cercanos al 8\% del PIB en años recientes.

En segundo lugar, la competencia entre el Banco Central y el Ministerio
de Hacienda en el mercado de deuda pública ha resultado en un aumento de
los costos de financiamiento, particularmente en el tramo medio de la
curva de tasas de interés. La falta de coordinación entre ambas
instituciones se manifiesta en diferenciales significativos de tasas
para instrumentos de similar vencimiento, llegando a superar los 350
puntos base en algunos casos. Esta divergencia en las tasas de
colocación refleja objetivos institucionales distintos: mientras el
Ministerio de Hacienda busca minimizar el costo del financiamiento
público, el Banco Central utiliza la emisión de deuda como instrumento
de política monetaria para controlar la liquidez y estabilizar el tipo
de cambio. \citep{oecd_mercado_2012}

El tercer aspecto crítico concierne a la evolución del déficit
cuasifiscal y sus implicaciones para las expectativas inflacionarias en
el marco de la curva de Phillips. La acumulación de pérdidas operativas
del Banco Central, que se ha intensificado desde la crisis de 2003-2004,
genera preocupaciones sobre la sostenibilidad de largo plazo de este
esquema de política monetaria. La literatura empírica sugiere que los
déficits cuasifiscales significativos pueden eventualmente resultar en
presiones inflacionarias, sea a través de la monetización directa del
déficit o mediante el deterioro de las expectativas de los agentes
económicos. \citep{cruz-rodriguez_deficit_2006}

Un cuarto elemento crítico radica en las limitaciones constitucionales
que enfrenta el Banco Central para financiar al gobierno central en el
mercado de bonos. Esta restricción, que solo puede ser levantada en
circunstancias excepcionales y bajo condiciones estrictas, reduce
significativamente los instrumentos disponibles para el control de la
depreciación del tipo de cambio.

El debate sobre la reforma de este esquema institucional ha generado dos
posiciones principales. Por un lado, algunos economistas, abogan por la
consolidación de la deuda gubernamental bajo el Ministerio de Hacienda,
argumentando que la gestión actual ha sido ineficiente y ha resultado en
un crecimiento significativo de la deuda del Banco Central, que
actualmente representa aproximadamente el 15\% del PIB. Esta posición
sugiere que la centralización del manejo de la deuda en un solo emisor
soberano permitiría mejorar las condiciones de emisión y reducir los
costos de financiamiento.

La posición contraria sostiene que, dada la diferente naturaleza de las
funciones de reacción de ambas instituciones, la unificación de la
emisión de deuda podría ser contraproducente para el logro de sus
respectivos objetivos. Además, argumentan que el Banco Central perdería
su instrumento más efectivo para controlar las presiones sobre el tipo
de cambio, lo cual podría comprometer la estabilidad macroeconómica en
un contexto de alta vulnerabilidad externa.

La determinación del régimen óptimo de coordinación económica representa
uno de los mayores desafíos de política económica que enfrenta la
República Dominicana actualmente. Esta problemática presenta dos
dimensiones fundamentales: por un lado, la necesidad de establecer
mecanismos efectivos de coordinación entre las autoridades monetarias y
financieras, y por otro, la identificación de los instrumentos más
adecuados para alcanzar los objetivos de política. La complejidad de
este reto radica en la necesidad de adoptar una perspectiva
multidimensional que considere los diferentes equilibrios económicos
implicados. El mecanismo actual de control de liquidez presenta un
dilema fundamental: ¿es preferible mantener una política con mayor costo
económico que ha demostrado efectividad en el control de las
expectativas cambiarias, o este enfoque representa un riesgo
insostenible para las finanzas del Banco Central en el largo plazo? La
resolución de esta disyuntiva requiere la consideración de múltiples
factores, incluyendo la prima de riesgo por intervención de liquidez, la
competencia en los tramos corto y medio de la curva de intereses, y las
implicaciones de la ``aritmética monetarista desagradable''. Como
resultado, la búsqueda de un marco institucional óptimo que balancee
estos diferentes objetivos y restricciones se mantiene como una pregunta
abierta en el diseño de la política económica dominicana.

\subsection{Propósito de la
Investigación}\label{propuxf3sito-de-la-investigaciuxf3n}

Este estudio contribuye a la literatura de política económica mediante
el análisis de los regímenes de coordinación óptimos entre la política
monetaria y fiscal en la República Dominicana, empleando un marco
teórico de juegos para una economía pequeña y abierta. La investigación
desarrolla un modelo de equilibrio general dinámico y estocástico (DSGE)
que incorpora las características fundamentales de la economía
dominicana, incluyendo una restricción intertemporal consolidada para
las instituciones gubernamentales y un mecanismo de transmisión del tipo
de cambio a la inflación. Este enfoque permite examinar las
interacciones estratégicas entre las autoridades monetarias y fiscales
bajo diferentes escenarios de coordinación, contribuyendo así a la
comprensión de las implicaciones del déficit cuasifiscal del Banco
Central.

La metodología propuesta se distingue por tres aspectos innovadores.
Primero, desarrolla un análisis basado en teoría de juegos del esquema
líder-seguidor entre las políticas monetaria y fiscal, específicamente
adaptado para una economía pequeña y abierta como la dominicana.
Segundo, implementa un conjunto de experimentos contrafactuales
utilizando parámetros estructurales del período 2012-2024, que
permitirán evaluar diferentes regímenes de coordinación y sus
implicaciones para el bienestar social. Tercero, incorpora un análisis
de sensibilidad global que examina la robustez de los resultados ante
variaciones en los parámetros estructurales del modelo, proporcionando
así una evaluación comprehensiva de la estabilidad y unicidad de los
equilibrios encontrados. Esta aproximación metodológica permite
identificar el régimen de coordinación que minimiza las pérdidas
sociales, ofreciendo recomendaciones de política económica basadas en
evidencia cuantitativa.

\subsection{Preguntas de
Investigación}\label{preguntas-de-investigaciuxf3n}

\begin{enumerate}
\def\labelenumi{\arabic{enumi}.}
\item
  ¿Cuál es el esquema óptimo de coordinación de políticas fiscales y
  monetarias en el contexto de la economía dominicana?
\item
  ¿Cuál es el dominio de los parámetros estructurales de forma que se
  asegura la estabilidad del regimen de coordinación óptimo?
\item
  ¿Qué reformas y/o instrumentos alternativos podrían ser necesarios
  para alcanzar el esquema de coordinación óptimo previamente definido?
\end{enumerate}

\section{Revisión Literaria}\label{revisiuxf3n-literaria}

\subsection{Marco teórico}\label{marco-teuxf3rico}

El análisis de la política monetaria se ha centrado históricamente en
responder dos preguntas fundamentales: los efectos de los cambios en la
postura monetaria sobre la economía y las circunstancias que llevan a un
banco central a modificar dicha postura. Esta segunda interrogante ha
cobrado especial relevancia en el contexto de las reglas de política
monetaria, particularmente en países que siguen esquemas de metas
explícitas de inflación.

La discusión sobre reglas versus discrecionalidad en política monetaria
tiene sus raíces en los trabajos seminales de \citep{kydland_rules_1977}
y \citep{barro_rules_1983}, quienes demostraron que la extrema
flexibilidad o discrecionalidad en el manejo monetario conduce a una
pérdida de credibilidad, afectando directamente el objetivo de
estabilización y generando un ``sesgo inflacionario''. Este hallazgo
fundamental llevó al desarrollo de un marco analítico basado en reglas
monetarias que pudieran servir como guía para las intervenciones de los
bancos centrales.

\citep{taylor_discretion_1993} marcó un hito en esta literatura al
proponer una regla simple que relaciona la tasa de interés nominal con
las desviaciones de la inflación respecto a su nivel objetivo y las
desviaciones del producto respecto a su nivel potencial. Su trabajo
demostró que las políticas monetarias centradas en estos objetivos son
más eficientes que aquellas que se ajustan con la oferta de dinero y el
tipo de cambio. \citep{svensson_inflation_1997}y
\citep{clarida_science_1999} posteriormente formalizaron los fundamentos
teóricos de la regla de Taylor y propusieron diversas modificaciones
sobre esta base de referencia.

La interacción entre políticas monetaria y fiscal introduce una
dimensión adicional al análisis. \citep{nordhaus_policy_1994} examinó
esta interacción desde un marco de teoría de juegos, encontrando que la
falta de coordinación entre autoridades puede resultar en equilibrios
subóptimos caracterizados por altas tasas de interés y déficits
presupuestarios. \citep{van_aarle_monetary_1995} profundizaron este
análisis en el contexto de la estabilización de la deuda, mientras que
estudiaron la necesidad de ajuste de la política monetaria para mantener
baja inflación en el marco de distintas estructuras de coordinación
entre hacedores de política.

El desarrollo de modelos de equilibrio general dinámico y estocástico
(DSGE) ha proporcionado un marco más riguroso para analizar estas
interacciones. El modelo seminal de \citep{gali_monetary_2005} para
economías pequeñas y abiertas incorpora rigideces nominales y
competencia monopolística en un marco de optimización intertemporal. La
estructura del modelo incluye hogares que maximizan su utilidad,
empresas que operan en un entorno de competencia monopolística con
precios rígidos siguiendo el esquema de \citep{calvo_staggered_1983}, y
autoridades monetarias y fiscales que implementan políticas económicas.

Este marco teórico permite analizar la transmisión de la política
monetaria a través de varios canales. La rigidez de precios de Calvo
genera efectos reales de la política monetaria y captura la persistencia
observada en la inflación. La apertura de la economía se modela mediante
un índice de precios al consumidor que incluye tanto bienes domésticos
como importados, con un mecanismo de pass-through del tipo de cambio a
la inflación, aspecto crucial para economías pequeñas y abiertas.

La literatura más reciente, ejemplificada por trabajos como
\citep{bartolomeo_fiscal-monetary_2005}, ha integrado la teoría de
juegos en este marco DSGE para analizar las interacciones estratégicas
entre autoridades monetarias y fiscales. Este enfoque permite examinar
diferentes escenarios de coordinación, desde equilibrios Nash
simultáneos hasta soluciones Stackelberg con distintos ordenamientos de
liderazgo, proporcionando insights sobre los regímenes de coordinación
óptimos para diferentes estructuras económicas.

\subsection{Antecedentes}\label{antecedentes}

La literatura sobre coordinación de políticas económicas en la República
Dominicana presenta un vacío significativo en cuanto al análisis de
interacciones estratégicas entre autoridades monetaria y fiscal mediante
teoría de juegos en un contexto de modelos DSGE. El referente
internacional más cercano a este tipo de análisis es el trabajo de
\citep{tetik_evaluation_2021} para la economía turca, que comparte
características similares como economía pequeña y abierta. Esta brecha
en la literatura dominicana representa una oportunidad importante para
expandir nuestra comprensión de los mecanismos de coordinación de
políticas económicas en el país.

\citep{tetik_evaluation_2021} aplica teoría de juegos al análisis de la
interacción entre autoridades monetaria y fiscal sobre el modelo
neokeynesiano de Galí. Mediante experimentos contrafactuales con datos
de la economía turca para 2006-2019, encuentra que el escenario que
minimiza la pérdida social ocurre cuando la autoridad monetaria actúa
como líder de Stackelberg. Sus resultados muestran patrones de respuesta
específicos ante choques exógenos, que varían según la naturaleza del
choque y la posición jerárquica de cada autoridad.

\citep{ramirez_modelo_2014} realiza la primera publicación de una
estimación de un modelo DSGE para la República Dominicana basado en el
marco de \citep{gali_monetary_2005}. Su modelo incorpora fricciones
nominales y persistencia en hábitos de consumo característicos de la
economía dominicana. La estimación bayesiana de parámetros estructurales
establece una base cuantitativa para el análisis de política monetaria
en el país.

\citep{perez_perez_nueva_2021} determina reglas de política monetaria
óptimas para la República Dominicana comparando especificaciones de
curvas de reacción. Su investigación indica que una regla de política
monetaria forward-looking que considera la inflación observada, la
inflación proyectada y la inercia de la tasa de interés genera mejores
resultados de bienestar social, con variaciones según los choques que
afectan a la economía.

La presente investigación se construye sobre estos antecedentes de
manera integral, combinando el marco metodológico de teoría de juegos
desarrollado por Tetik con las estimaciones de parámetros estructurales
de Ramírez y los hallazgos sobre reglas monetarias óptimas de Pérez.
Esta síntesis permite desarrollar un marco analítico robusto que examina
por primera vez las interacciones estratégicas entre autoridades
monetaria y fiscal en la República Dominicana.

\section{Estructura del Modelo}\label{estructura-del-modelo}

\subsection{Un modelo DSGE neokeynesiano para la economía
dominicana}\label{un-modelo-dsge-neokeynesiano-para-la-economuxeda-dominicana}

\subsection{Derivación de las curvas de reacción bajo un regimen de
independencia
total.}\label{derivaciuxf3n-de-las-curvas-de-reacciuxf3n-bajo-un-regimen-de-independencia-total.}

\subsection{Derivación de las curvas de reacción bajo un regimen de
dominancia
fiscal}\label{derivaciuxf3n-de-las-curvas-de-reacciuxf3n-bajo-un-regimen-de-dominancia-fiscal}

\subsection{Derivación de la curva de reacción bajo un regimen de
dominancia
monetaria}\label{derivaciuxf3n-de-la-curva-de-reacciuxf3n-bajo-un-regimen-de-dominancia-monetaria}

\subsection{Calibración paramétrica para la simulación
dinámica}\label{calibraciuxf3n-paramuxe9trica-para-la-simulaciuxf3n-dinuxe1mica}

\section{Resultados}\label{resultados}

\subsection{Análisis de perdidas sociales: Determinación del regimen de
interacción
óptimo}\label{anuxe1lisis-de-perdidas-sociales-determinaciuxf3n-del-regimen-de-interacciuxf3n-uxf3ptimo}

\subsection{Análisis de los choques estructurales: Función de
impulso-reacción}\label{anuxe1lisis-de-los-choques-estructurales-funciuxf3n-de-impulso-reacciuxf3n}

\subsection{Análisis de sensibilidad: Evaluación de la estabilidad
paramétrica por filtro de Monte
Carlo}\label{anuxe1lisis-de-sensibilidad-evaluaciuxf3n-de-la-estabilidad-paramuxe9trica-por-filtro-de-monte-carlo}

\section{Conclusiones y
Recomendaciones}\label{conclusiones-y-recomendaciones}

\section{Referencias}\label{referencias}

\renewcommand{\bibsection}{}
\bibliography{references.bib}





\end{document}
