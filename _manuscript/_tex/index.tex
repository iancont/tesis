% Options for packages loaded elsewhere
\PassOptionsToPackage{unicode}{hyperref}
\PassOptionsToPackage{hyphens}{url}
\PassOptionsToPackage{dvipsnames,svgnames,x11names}{xcolor}
%
\documentclass[
  authoryear,
  review]{elsarticle}

\usepackage{amsmath,amssymb}
\usepackage{iftex}
\ifPDFTeX
  \usepackage[T1]{fontenc}
  \usepackage[utf8]{inputenc}
  \usepackage{textcomp} % provide euro and other symbols
\else % if luatex or xetex
  \usepackage{unicode-math}
  \defaultfontfeatures{Scale=MatchLowercase}
  \defaultfontfeatures[\rmfamily]{Ligatures=TeX,Scale=1}
\fi
\usepackage{lmodern}
\ifPDFTeX\else  
    % xetex/luatex font selection
\fi
% Use upquote if available, for straight quotes in verbatim environments
\IfFileExists{upquote.sty}{\usepackage{upquote}}{}
\IfFileExists{microtype.sty}{% use microtype if available
  \usepackage[]{microtype}
  \UseMicrotypeSet[protrusion]{basicmath} % disable protrusion for tt fonts
}{}
\makeatletter
\@ifundefined{KOMAClassName}{% if non-KOMA class
  \IfFileExists{parskip.sty}{%
    \usepackage{parskip}
  }{% else
    \setlength{\parindent}{0pt}
    \setlength{\parskip}{6pt plus 2pt minus 1pt}}
}{% if KOMA class
  \KOMAoptions{parskip=half}}
\makeatother
\usepackage{xcolor}
\setlength{\emergencystretch}{3em} % prevent overfull lines
\setcounter{secnumdepth}{5}
% Make \paragraph and \subparagraph free-standing
\makeatletter
\ifx\paragraph\undefined\else
  \let\oldparagraph\paragraph
  \renewcommand{\paragraph}{
    \@ifstar
      \xxxParagraphStar
      \xxxParagraphNoStar
  }
  \newcommand{\xxxParagraphStar}[1]{\oldparagraph*{#1}\mbox{}}
  \newcommand{\xxxParagraphNoStar}[1]{\oldparagraph{#1}\mbox{}}
\fi
\ifx\subparagraph\undefined\else
  \let\oldsubparagraph\subparagraph
  \renewcommand{\subparagraph}{
    \@ifstar
      \xxxSubParagraphStar
      \xxxSubParagraphNoStar
  }
  \newcommand{\xxxSubParagraphStar}[1]{\oldsubparagraph*{#1}\mbox{}}
  \newcommand{\xxxSubParagraphNoStar}[1]{\oldsubparagraph{#1}\mbox{}}
\fi
\makeatother


\providecommand{\tightlist}{%
  \setlength{\itemsep}{0pt}\setlength{\parskip}{0pt}}\usepackage{longtable,booktabs,array}
\usepackage{calc} % for calculating minipage widths
% Correct order of tables after \paragraph or \subparagraph
\usepackage{etoolbox}
\makeatletter
\patchcmd\longtable{\par}{\if@noskipsec\mbox{}\fi\par}{}{}
\makeatother
% Allow footnotes in longtable head/foot
\IfFileExists{footnotehyper.sty}{\usepackage{footnotehyper}}{\usepackage{footnote}}
\makesavenoteenv{longtable}
\usepackage{graphicx}
\makeatletter
\newsavebox\pandoc@box
\newcommand*\pandocbounded[1]{% scales image to fit in text height/width
  \sbox\pandoc@box{#1}%
  \Gscale@div\@tempa{\textheight}{\dimexpr\ht\pandoc@box+\dp\pandoc@box\relax}%
  \Gscale@div\@tempb{\linewidth}{\wd\pandoc@box}%
  \ifdim\@tempb\p@<\@tempa\p@\let\@tempa\@tempb\fi% select the smaller of both
  \ifdim\@tempa\p@<\p@\scalebox{\@tempa}{\usebox\pandoc@box}%
  \else\usebox{\pandoc@box}%
  \fi%
}
% Set default figure placement to htbp
\def\fps@figure{htbp}
\makeatother

\makeatletter
\@ifpackageloaded{caption}{}{\usepackage{caption}}
\AtBeginDocument{%
\ifdefined\contentsname
  \renewcommand*\contentsname{Tabla de contenidos}
\else
  \newcommand\contentsname{Tabla de contenidos}
\fi
\ifdefined\listfigurename
  \renewcommand*\listfigurename{Listado de Figuras}
\else
  \newcommand\listfigurename{Listado de Figuras}
\fi
\ifdefined\listtablename
  \renewcommand*\listtablename{Listado de Tablas}
\else
  \newcommand\listtablename{Listado de Tablas}
\fi
\ifdefined\figurename
  \renewcommand*\figurename{Figura}
\else
  \newcommand\figurename{Figura}
\fi
\ifdefined\tablename
  \renewcommand*\tablename{Tabla}
\else
  \newcommand\tablename{Tabla}
\fi
}
\@ifpackageloaded{float}{}{\usepackage{float}}
\floatstyle{ruled}
\@ifundefined{c@chapter}{\newfloat{codelisting}{h}{lop}}{\newfloat{codelisting}{h}{lop}[chapter]}
\floatname{codelisting}{Listado}
\newcommand*\listoflistings{\listof{codelisting}{Listado de Listados}}
\makeatother
\makeatletter
\makeatother
\makeatletter
\@ifpackageloaded{caption}{}{\usepackage{caption}}
\@ifpackageloaded{subcaption}{}{\usepackage{subcaption}}
\makeatother
\journal{Journal Name}
\makeatletter
\@ifpackageloaded{fontspec}{}{\usepackage{fontspec}}
\makeatother
\makeatletter
\@ifpackageloaded{draftwatermark}{}{\usepackage{draftwatermark}}
\makeatother
\makeatletter
\@ifpackageloaded{xcolor}{}{\usepackage{xcolor}}
\makeatother
\makeatletter
\@ifpackageloaded{forloop}{}{\usepackage{forloop}}
\makeatother
    \definecolor{watermark}{HTML}{000000}
    
    \newcounter{watermarkrow}
    \newcounter{watermarkcol}

    \DraftwatermarkOptions{
      text={
        \begin{tabular}{c}
          \forloop{watermarkrow}{0}{\value{watermarkrow} < 50}{
            \forloop{watermarkcol}{0}{\value{watermarkcol} < 10}{
              { Borrador}\hspace{4.000000em}
            }
            \\[4.000000em]
          }
        \end{tabular}
      },
      fontsize=1.500000em,
      angle=15.000000,
      color=watermark!20
    }
    

\ifLuaTeX
\usepackage[bidi=basic]{babel}
\else
\usepackage[bidi=default]{babel}
\fi
\babelprovide[main,import]{spanish}
% get rid of language-specific shorthands (see #6817):
\let\LanguageShortHands\languageshorthands
\def\languageshorthands#1{}
\usepackage[]{natbib}
\bibliographystyle{elsarticle-harv}
\nocite{*}
\usepackage{bookmark}

\IfFileExists{xurl.sty}{\usepackage{xurl}}{} % add URL line breaks if available
\urlstyle{same} % disable monospaced font for URLs
\hypersetup{
  pdftitle={Short Paper},
  pdfauthor={Ian Contreras},
  pdflang={es},
  pdfkeywords={keyword1, keyword2},
  colorlinks=true,
  linkcolor={blue},
  filecolor={Maroon},
  citecolor={Blue},
  urlcolor={Blue},
  pdfcreator={LaTeX via pandoc}}


\setlength{\parindent}{6pt}
\begin{document}

\begin{frontmatter}
\title{Short Paper \\\large{A Short Subtitle} }
\author[1]{Ian Contreras%
\corref{cor1}%
\fnref{fn1}}


\affiliation[1]{organization={Instituto Tecnológico de Santo
Domingo, Economía y Negocios},city={Santo Domingo},country={República
Dominicana},countrysep={,},postcodesep={}}

\cortext[cor1]{Corresponding author}
\fntext[fn1]{Lorem Ipsum}
        
\begin{abstract}
This is the abstract. Lorem ipsum dolor sit amet, consectetur adipiscing
elit. Vestibulum augue turpis, dictum non malesuada a, volutpat eget
velit. Nam placerat turpis purus, eu tristique ex tincidunt et. Mauris
sed augue eget turpis ultrices tincidunt. Sed et mi in leo porta
egestas. Aliquam non laoreet velit. Nunc quis ex vitae eros aliquet
auctor nec ac libero. Duis laoreet sapien eu mi luctus, in bibendum leo
molestie. Sed hendrerit diam diam, ac dapibus nisl volutpat vitae.
Aliquam bibendum varius libero, eu efficitur justo rutrum at. Sed at
tempus elit.
\end{abstract}





\begin{keyword}
    keyword1 \sep 
    keyword2
\end{keyword}
\end{frontmatter}
    
\renewcommand*\contentsname{Tabla de contenidos}
{
\hypersetup{linkcolor=}
\setcounter{tocdepth}{3}
\tableofcontents
}

\section{Introducción}\label{introducciuxf3n}

\subsection{Planteamiento del
Problema}\label{planteamiento-del-problema}

La coexistencia de dos emisores de deuda gubernamental se ha convertido
en uno de los puntos más cuestionados en la economía dominicana de la
última década. La principal crítica radica en que el Banco Central
realiza una gestión ineficiente de su deuda, reflejada en la evolución
negativa de su patrimonio. No obstante, dado que el Banco Central opera
bajo un régimen de flotación sucia, en el cual la emisión de deuda
funciona como herramienta de control de liquidez (quantitative
tightening), resulta inapropiado evaluar su manejo de pasivos desde una
óptica meramente contable.

El origen de esta dualidad en la emisión de deuda pública se remonta a
la crisis financiera de 2003-2004, cuando el colapso de tres importantes
instituciones bancarias (Baninter, Bancrédito y Banco Mercantil) provocó
una intervención masiva del Banco Central para evitar un riesgo
sistémico. El rescate bancario, que representó aproximadamente el 20.3\%
del PIB, resultó en una emisión de RD\$109,150 millones en instrumentos
de deuda por parte del Banco Central. Esta intervención, si bien
necesaria para mantener la estabilidad del sistema financiero, generó un
deterioro significativo en el balance del banco central y dio origen a
un déficit cuasifiscal que persiste hasta la actualidad, condicionando
tanto el manejo de la política monetaria como fiscal.
\citep{oecd_mercado_2012}

La problemática actual presenta tres dimensiones críticas
interrelacionadas. En primer lugar, existen desafíos significativos en
la coordinación de las políticas monetaria y fiscal en un contexto de
vulnerabilidad de la balanza de pagos. República Dominicana, como
importador neto de commodities, enfrenta limitaciones estructurales para
contrarrestar los choques externos en los precios de importación, lo que
genera presiones recurrentes sobre el tipo de cambio y condiciona la
efectividad de la política monetaria. Esta situación se refleja en un
déficit persistente en cuenta corriente, que ha alcanzado niveles
cercanos al 8\% del PIB en años recientes.

En segundo lugar, la competencia entre el Banco Central y el Ministerio
de Hacienda en el mercado de deuda pública ha resultado en un aumento de
los costos de financiamiento, particularmente en el tramo medio de la
curva de tasas de interés. La falta de coordinación entre ambas
instituciones se manifiesta en diferenciales significativos de tasas
para instrumentos de similar vencimiento, llegando a superar los 350
puntos base en algunos casos. Esta divergencia en las tasas de
colocación refleja objetivos institucionales distintos: mientras el
Ministerio de Hacienda busca minimizar el costo del financiamiento
público, el Banco Central utiliza la emisión de deuda como instrumento
de política monetaria para controlar la liquidez y estabilizar el tipo
de cambio. \citep{oecd_mercado_2012}

El tercer aspecto crítico concierne a la evolución del déficit
cuasifiscal y sus implicaciones para las expectativas inflacionarias en
el marco de la curva de Phillips. La acumulación de pérdidas operativas
del Banco Central, que se ha intensificado desde la crisis de 2003-2004,
genera preocupaciones sobre la sostenibilidad de largo plazo de este
esquema de política monetaria. La literatura empírica sugiere que los
déficits cuasifiscales significativos pueden eventualmente resultar en
presiones inflacionarias, sea a través de la monetización directa del
déficit o mediante el deterioro de las expectativas de los agentes
económicos. \citep{cruz-rodriguez_deficit_2006}

El debate sobre la reforma de este esquema institucional ha generado dos
posiciones principales. Por un lado, algunos economistas, abogan por la
consolidación de la deuda gubernamental bajo el Ministerio de Hacienda,
argumentando que la gestión actual ha sido ineficiente y ha resultado en
un crecimiento significativo de la deuda del Banco Central, que
actualmente representa aproximadamente el 15\% del PIB. Esta posición
sugiere que la centralización del manejo de la deuda en un solo emisor
soberano permitiría mejorar las condiciones de emisión y reducir los
costos de financiamiento.

La posición contraria sostiene que, dada la diferente naturaleza de las
funciones de reacción de ambas instituciones, la unificación de la
emisión de deuda podría ser contraproducente para el logro de sus
respectivos objetivos. Además, argumentan que el Banco Central perdería
su instrumento más efectivo para controlar las presiones sobre el tipo
de cambio, lo cual podría comprometer la estabilidad macroeconómica en
un contexto de alta vulnerabilidad externa.

La determinación del régimen óptimo de coordinación económica representa
uno de los mayores desafíos de política económica que enfrenta la
República Dominicana actualmente. Esta problemática presenta dos
dimensiones fundamentales: por un lado, la necesidad de establecer
mecanismos efectivos de coordinación entre las autoridades monetarias y
financieras, y por otro, la identificación de los instrumentos más
adecuados para alcanzar los objetivos de política. La complejidad de
este reto radica en la necesidad de adoptar una perspectiva
multidimensional que considere los diferentes equilibrios económicos
implicados. El mecanismo actual de control de liquidez presenta un
dilema fundamental: ¿es preferible mantener una política con mayor costo
económico que ha demostrado efectividad en el control de las
expectativas cambiarias, o este enfoque representa un riesgo
insostenible para las finanzas del Banco Central en el largo plazo? La
resolución de esta disyuntiva requiere la consideración de múltiples
factores, incluyendo la prima de riesgo por intervención de liquidez, la
competencia en los tramos corto y medio de la curva de intereses, y las
implicaciones de la ``aritmética monetarista desagradable''. Como
resultado, la búsqueda de un marco institucional óptimo que balancee
estos diferentes objetivos y restricciones se mantiene como una pregunta
abierta en el diseño de la política económica dominicana.

\subsection{Propósito de la
Investigación}\label{propuxf3sito-de-la-investigaciuxf3n}

\subsection{Preguntas de
Investigación}\label{preguntas-de-investigaciuxf3n}

\section{Revisión Literaria}\label{revisiuxf3n-literaria}

\section{Metodología}\label{metodologuxeda}

\subsection{Un modelo DSGE neokeynesiano para la economía
dominicana}\label{un-modelo-dsge-neokeynesiano-para-la-economuxeda-dominicana}

\subsection{Derivación de las curvas de reacción bajo un regimen de
independencia
total.}\label{derivaciuxf3n-de-las-curvas-de-reacciuxf3n-bajo-un-regimen-de-independencia-total.}

\subsection{Derivación de las curvas de reacción bajo un regimen de
dominancia
fiscal}\label{derivaciuxf3n-de-las-curvas-de-reacciuxf3n-bajo-un-regimen-de-dominancia-fiscal}

\subsection{Derivación de la curva de reacción bajo un regimen de
dominancia
monetaria}\label{derivaciuxf3n-de-la-curva-de-reacciuxf3n-bajo-un-regimen-de-dominancia-monetaria}

\subsection{Calibración paramétrica para la simulación
dinámica}\label{calibraciuxf3n-paramuxe9trica-para-la-simulaciuxf3n-dinuxe1mica}

\section{Resultados}\label{resultados}

\subsection{Análisis de perdidas sociales: Determinación del regimen de
interacción
óptimo}\label{anuxe1lisis-de-perdidas-sociales-determinaciuxf3n-del-regimen-de-interacciuxf3n-uxf3ptimo}

\subsection{Análisis de los choques estructurales: Función de
impulso-reacción}\label{anuxe1lisis-de-los-choques-estructurales-funciuxf3n-de-impulso-reacciuxf3n}

\subsection{Análisis de sensibilidad: Evaluación de la estabilidad
paramétrica por filtro de Monte
Carlo}\label{anuxe1lisis-de-sensibilidad-evaluaciuxf3n-de-la-estabilidad-paramuxe9trica-por-filtro-de-monte-carlo}

\section{Conclusiones y
Recomendaciones}\label{conclusiones-y-recomendaciones}

\section{Referencias}\label{referencias}

\renewcommand{\bibsection}{}
\bibliography{references.bib}





\end{document}
